%%% Research Diary - Entry
%%% Template by Mikhail Klassen, April 2013
%%% 
\documentclass[11pt,letterpaper]{article}

\newcommand{\workingDate}{\today}
\newcommand{\userName}{EMCH 561}
\newcommand{\institution}{University of South Carolina}
\usepackage{siunitx}
\usepackage{Proposal}
\usepackage{hyperref}
\usepackage{indentfirst}
\usepackage[backend=biber, style=ieee]{biblatex}
\addbibresource{references.bib}

% To add your university logo to the upper right, simply
% upload a file named "logo.png" using the files menu above.

\begin{document}
\univlogo

\title{EMCH 561 Team 4 Proposal}

{\noindent\Huge EMCH 561 Project Proposal}\\[5mm]
By Jack Hannum, Claud Boyd, Jarret Peskar, Braden Priddy, and Zhymir Thompson


\section*{Introduction}

3-phase Electric motors can be driven by an inverter, which both supplies the power to control the motor, and provides the means to control the motor's speed and torque. 
For speed control of an induction motor, the variable under control are the phase voltages.
Typically, feedback control requires the measurement of the control variable, which is subtracted from a reference, creating an error signal that can be fed to a controller.
In the motor control context, however, the phase voltages typically are not measured due to the expense and complexity of doing so. 
Rotor speed is a much more easily measured quantity, and this is typically what is measured in practical applications of electric motors. 
To generate the error signal required for feedback control of the motor, a model relating rotor speed to phase voltage can be employed.
The creation of a data-driven model between rotor speed and phase voltage is the objective of this project. 

\section{Data}

Marius Stender, Oliver Wallscheid, and Joachim Bocker at Paderborn University in Germany have created a dataset by adding instrumentation to an inverter as it drives a 3-phase induction motor \cite{Stender2020-qi}\cite{Stender_undated-rx}. The data is available as a Comma Separated Values (CSV) file with 26 dimensions:
\begin{enumerate}
	\item Motor Speed [RPM]
	\item DC-link Voltage [V]
	\item Previous Step DC-link Voltage [V]
	\item Two Steps Previous Step DC-link Voltage [V]
	\item Three Steps Previous Step DC-link Voltage [V]
	\item Phase-A Current [A]
	\item Phase-B Current [A]
	\item Phase-C Current [A]
	\item Previous Step Phase-A Current [A]
	\item Previous Step Phase-B Current [A]
	\item Previous Step Phase-C Current [A]
	\item Two steps Previous Phase-A Current [A]
	\item Two steps Previous Phase-B Current [A]
	\item Two steps Previous Phase-C Current [A]
	\item Three steps Previous Phase-A Current [A]
	\item Three steps Previous Phase-B Current [A]
	\item Three steps Previous Phase-C Current [A]
	\item Two steps Previous Phase-A Duty Cycle [0-1]
	\item Two steps Previous Phase-B Duty Cycle [0-1]
	\item Two steps Previous Phase-C Duty Cycle [0-1]
	\item Three steps Previous Phase-A Duty Cycle [0-1]
	\item Three steps Previous Phase-B Duty Cycle [0-1]
	\item Three steps Previous Phase-C Duty Cycle [0-1]
	\item Previous Step Phase-A Voltage [V]
	\item Previous Step Phase-B Voltage [V]
	\item Previous Step Phase-C Voltage [V]
\end{enumerate}

The data was sampled at \SI{10}{\kHz}, which for a motor spinning at approximately \SI{3000}{RPM}, or \SI{50}{\Hz}, should enable complete recreation of motor dynamics, and switching converter dynamics up to \SI{5}{\kHz} per Nyquist's theorem.
The inverter switched at \SI{10}{\kHz}, which means that some converter dynamics will be lost from sampling; this should not be an issue, as the model to be created should not be dependent on accurate modeling of individual switching periods. 

\section{Type of Machine Learning to be Applied}

The type of Machine Learning to be applied to this problem is a supervised regression algorithm, since a labeled training set has been provided, and a number (Phase-A voltage, which for a balanced 3-phase motor can determine the voltages of the remaining phases) is to be predicted. 

\section{Training Type}

This is a batch training problem, as all available data will be used to train the regression model, and the dataset is sufficiently small (\SI{102}{MB}) to fit in memory without requiring Mini-batching or stochastic regression.

\section{Instance or Model Based Problem}

This is a model based problem, as an optimal model will be determined from the training data, and new data will be predicted using this model.

\section{Algorithm Type}

The algorithm to be applied is a regression algorithm, potentially linear regression using gradient descent.

\printbibliography

\end{document}